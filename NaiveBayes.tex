\documentclass[]{article}
\usepackage{lmodern}
\usepackage{amssymb,amsmath}
\usepackage{ifxetex,ifluatex}
\usepackage{fixltx2e} % provides \textsubscript
\ifnum 0\ifxetex 1\fi\ifluatex 1\fi=0 % if pdftex
  \usepackage[T1]{fontenc}
  \usepackage[utf8]{inputenc}
\else % if luatex or xelatex
  \ifxetex
    \usepackage{mathspec}
  \else
    \usepackage{fontspec}
  \fi
  \defaultfontfeatures{Ligatures=TeX,Scale=MatchLowercase}
\fi
% use upquote if available, for straight quotes in verbatim environments
\IfFileExists{upquote.sty}{\usepackage{upquote}}{}
% use microtype if available
\IfFileExists{microtype.sty}{%
\usepackage{microtype}
\UseMicrotypeSet[protrusion]{basicmath} % disable protrusion for tt fonts
}{}
\usepackage[margin=1in]{geometry}
\usepackage{hyperref}
\hypersetup{unicode=true,
            pdftitle={Naive Bayes},
            pdfborder={0 0 0},
            breaklinks=true}
\urlstyle{same}  % don't use monospace font for urls
\usepackage{color}
\usepackage{fancyvrb}
\newcommand{\VerbBar}{|}
\newcommand{\VERB}{\Verb[commandchars=\\\{\}]}
\DefineVerbatimEnvironment{Highlighting}{Verbatim}{commandchars=\\\{\}}
% Add ',fontsize=\small' for more characters per line
\usepackage{framed}
\definecolor{shadecolor}{RGB}{248,248,248}
\newenvironment{Shaded}{\begin{snugshade}}{\end{snugshade}}
\newcommand{\KeywordTok}[1]{\textcolor[rgb]{0.13,0.29,0.53}{\textbf{#1}}}
\newcommand{\DataTypeTok}[1]{\textcolor[rgb]{0.13,0.29,0.53}{#1}}
\newcommand{\DecValTok}[1]{\textcolor[rgb]{0.00,0.00,0.81}{#1}}
\newcommand{\BaseNTok}[1]{\textcolor[rgb]{0.00,0.00,0.81}{#1}}
\newcommand{\FloatTok}[1]{\textcolor[rgb]{0.00,0.00,0.81}{#1}}
\newcommand{\ConstantTok}[1]{\textcolor[rgb]{0.00,0.00,0.00}{#1}}
\newcommand{\CharTok}[1]{\textcolor[rgb]{0.31,0.60,0.02}{#1}}
\newcommand{\SpecialCharTok}[1]{\textcolor[rgb]{0.00,0.00,0.00}{#1}}
\newcommand{\StringTok}[1]{\textcolor[rgb]{0.31,0.60,0.02}{#1}}
\newcommand{\VerbatimStringTok}[1]{\textcolor[rgb]{0.31,0.60,0.02}{#1}}
\newcommand{\SpecialStringTok}[1]{\textcolor[rgb]{0.31,0.60,0.02}{#1}}
\newcommand{\ImportTok}[1]{#1}
\newcommand{\CommentTok}[1]{\textcolor[rgb]{0.56,0.35,0.01}{\textit{#1}}}
\newcommand{\DocumentationTok}[1]{\textcolor[rgb]{0.56,0.35,0.01}{\textbf{\textit{#1}}}}
\newcommand{\AnnotationTok}[1]{\textcolor[rgb]{0.56,0.35,0.01}{\textbf{\textit{#1}}}}
\newcommand{\CommentVarTok}[1]{\textcolor[rgb]{0.56,0.35,0.01}{\textbf{\textit{#1}}}}
\newcommand{\OtherTok}[1]{\textcolor[rgb]{0.56,0.35,0.01}{#1}}
\newcommand{\FunctionTok}[1]{\textcolor[rgb]{0.00,0.00,0.00}{#1}}
\newcommand{\VariableTok}[1]{\textcolor[rgb]{0.00,0.00,0.00}{#1}}
\newcommand{\ControlFlowTok}[1]{\textcolor[rgb]{0.13,0.29,0.53}{\textbf{#1}}}
\newcommand{\OperatorTok}[1]{\textcolor[rgb]{0.81,0.36,0.00}{\textbf{#1}}}
\newcommand{\BuiltInTok}[1]{#1}
\newcommand{\ExtensionTok}[1]{#1}
\newcommand{\PreprocessorTok}[1]{\textcolor[rgb]{0.56,0.35,0.01}{\textit{#1}}}
\newcommand{\AttributeTok}[1]{\textcolor[rgb]{0.77,0.63,0.00}{#1}}
\newcommand{\RegionMarkerTok}[1]{#1}
\newcommand{\InformationTok}[1]{\textcolor[rgb]{0.56,0.35,0.01}{\textbf{\textit{#1}}}}
\newcommand{\WarningTok}[1]{\textcolor[rgb]{0.56,0.35,0.01}{\textbf{\textit{#1}}}}
\newcommand{\AlertTok}[1]{\textcolor[rgb]{0.94,0.16,0.16}{#1}}
\newcommand{\ErrorTok}[1]{\textcolor[rgb]{0.64,0.00,0.00}{\textbf{#1}}}
\newcommand{\NormalTok}[1]{#1}
\usepackage{graphicx,grffile}
\makeatletter
\def\maxwidth{\ifdim\Gin@nat@width>\linewidth\linewidth\else\Gin@nat@width\fi}
\def\maxheight{\ifdim\Gin@nat@height>\textheight\textheight\else\Gin@nat@height\fi}
\makeatother
% Scale images if necessary, so that they will not overflow the page
% margins by default, and it is still possible to overwrite the defaults
% using explicit options in \includegraphics[width, height, ...]{}
\setkeys{Gin}{width=\maxwidth,height=\maxheight,keepaspectratio}
\IfFileExists{parskip.sty}{%
\usepackage{parskip}
}{% else
\setlength{\parindent}{0pt}
\setlength{\parskip}{6pt plus 2pt minus 1pt}
}
\setlength{\emergencystretch}{3em}  % prevent overfull lines
\providecommand{\tightlist}{%
  \setlength{\itemsep}{0pt}\setlength{\parskip}{0pt}}
\setcounter{secnumdepth}{0}
% Redefines (sub)paragraphs to behave more like sections
\ifx\paragraph\undefined\else
\let\oldparagraph\paragraph
\renewcommand{\paragraph}[1]{\oldparagraph{#1}\mbox{}}
\fi
\ifx\subparagraph\undefined\else
\let\oldsubparagraph\subparagraph
\renewcommand{\subparagraph}[1]{\oldsubparagraph{#1}\mbox{}}
\fi

%%% Use protect on footnotes to avoid problems with footnotes in titles
\let\rmarkdownfootnote\footnote%
\def\footnote{\protect\rmarkdownfootnote}

%%% Change title format to be more compact
\usepackage{titling}

% Create subtitle command for use in maketitle
\newcommand{\subtitle}[1]{
  \posttitle{
    \begin{center}\large#1\end{center}
    }
}

\setlength{\droptitle}{-2em}
  \title{Naive Bayes}
  \pretitle{\vspace{\droptitle}\centering\huge}
  \posttitle{\par}
  \author{}
  \preauthor{}\postauthor{}
  \date{}
  \predate{}\postdate{}


\begin{document}
\maketitle

\subsubsection{Ejemplo de Naive Bayes hecho en R sin utilizar librerias
externas}\label{ejemplo-de-naive-bayes-hecho-en-r-sin-utilizar-librerias-externas}

\begin{Shaded}
\begin{Highlighting}[]
\KeywordTok{library}\NormalTok{(dplyr)}
\end{Highlighting}
\end{Shaded}

\begin{verbatim}
## 
## Attaching package: 'dplyr'
\end{verbatim}

\begin{verbatim}
## The following objects are masked from 'package:stats':
## 
##     filter, lag
\end{verbatim}

\begin{verbatim}
## The following objects are masked from 'package:base':
## 
##     intersect, setdiff, setequal, union
\end{verbatim}

\begin{Shaded}
\begin{Highlighting}[]
\CommentTok{# Creamos un Data frame con las columnas:}
\CommentTok{# Sexo, Altura, Peso, Tamanio_pie }
\NormalTok{data <-}\StringTok{  }\KeywordTok{data.frame}\NormalTok{(}\DataTypeTok{sexo =} \KeywordTok{c}\NormalTok{(}\StringTok{'hombre'}\NormalTok{,}\StringTok{'hombre'}\NormalTok{,}\StringTok{'hombre'}\NormalTok{,}\StringTok{'hombre'}\NormalTok{,}
                          \StringTok{'mujer'}\NormalTok{,}\StringTok{'mujer'}\NormalTok{,}\StringTok{'mujer'}\NormalTok{,}\StringTok{'mujer'}\NormalTok{),}
                  \DataTypeTok{altura=}\KeywordTok{c}\NormalTok{(}\DecValTok{6}\NormalTok{,}\FloatTok{5.92}\NormalTok{,}\FloatTok{5.58}\NormalTok{,}\FloatTok{5.92}\NormalTok{,}\DecValTok{5}\NormalTok{,}\FloatTok{5.5}\NormalTok{,}\FloatTok{5.42}\NormalTok{,}\FloatTok{5.75}\NormalTok{),}
                  \DataTypeTok{peso=}\KeywordTok{c}\NormalTok{(}\DecValTok{180}\NormalTok{,}\DecValTok{190}\NormalTok{,}\DecValTok{170}\NormalTok{,}\DecValTok{165}\NormalTok{,}\DecValTok{100}\NormalTok{,}\DecValTok{150}\NormalTok{,}\DecValTok{130}\NormalTok{,}\DecValTok{150}\NormalTok{),}
                  \DataTypeTok{tamanio_pie=}\KeywordTok{c}\NormalTok{(}\DecValTok{12}\NormalTok{,}\DecValTok{11}\NormalTok{,}\DecValTok{12}\NormalTok{,}\DecValTok{10}\NormalTok{,}\DecValTok{6}\NormalTok{,}\DecValTok{8}\NormalTok{,}\DecValTok{7}\NormalTok{,}\DecValTok{9}\NormalTok{))}
\end{Highlighting}
\end{Shaded}

\begin{Shaded}
\begin{Highlighting}[]
\NormalTok{data}
\end{Highlighting}
\end{Shaded}

\begin{verbatim}
##     sexo altura peso tamanio_pie
## 1 hombre   6.00  180          12
## 2 hombre   5.92  190          11
## 3 hombre   5.58  170          12
## 4 hombre   5.92  165          10
## 5  mujer   5.00  100           6
## 6  mujer   5.50  150           8
## 7  mujer   5.42  130           7
## 8  mujer   5.75  150           9
\end{verbatim}

\begin{Shaded}
\begin{Highlighting}[]
\CommentTok{# Creamos un DataFrame para guardar los valores de que queremos probar.}
\NormalTok{persona =}\StringTok{ }\KeywordTok{data.frame}\NormalTok{(}\DataTypeTok{altura=}\KeywordTok{c}\NormalTok{(}\DecValTok{6}\NormalTok{),}
                     \DataTypeTok{peso=}\KeywordTok{c}\NormalTok{(}\DecValTok{130}\NormalTok{),}
                     \DataTypeTok{tamanio_pies=}\KeywordTok{c}\NormalTok{(}\DecValTok{8}\NormalTok{))}
\NormalTok{persona}
\end{Highlighting}
\end{Shaded}

\begin{verbatim}
##   altura peso tamanio_pies
## 1      6  130            8
\end{verbatim}

\begin{Shaded}
\begin{Highlighting}[]
\CommentTok{#  Numero de hombres}
\NormalTok{n_hombres <-}\StringTok{ }\KeywordTok{nrow}\NormalTok{(}\KeywordTok{subset}\NormalTok{(data, data}\OperatorTok{$}\NormalTok{sexo }\OperatorTok{==}\StringTok{ 'hombre'}\NormalTok{))}
\KeywordTok{print}\NormalTok{(n_hombres)}
\end{Highlighting}
\end{Shaded}

\begin{verbatim}
## [1] 4
\end{verbatim}

\begin{Shaded}
\begin{Highlighting}[]
\CommentTok{# Numero de mujeres}
\NormalTok{n_mujeres <-}\StringTok{ }\KeywordTok{nrow}\NormalTok{(}\KeywordTok{subset}\NormalTok{(data, data}\OperatorTok{$}\NormalTok{sexo }\OperatorTok{==}\StringTok{ 'mujer'}\NormalTok{))}
\KeywordTok{print}\NormalTok{(n_mujeres)}
\end{Highlighting}
\end{Shaded}

\begin{verbatim}
## [1] 4
\end{verbatim}

\begin{Shaded}
\begin{Highlighting}[]
\CommentTok{# Numero de registros}
\NormalTok{n_registros=}\KeywordTok{nrow}\NormalTok{(data)}
\KeywordTok{print}\NormalTok{(n_registros)}
\end{Highlighting}
\end{Shaded}

\begin{verbatim}
## [1] 8
\end{verbatim}

\subsubsection{Calculamos las probailidades
previas}\label{calculamos-las-probailidades-previas}

\begin{Shaded}
\begin{Highlighting}[]
\CommentTok{# Probabilidad de hombres}
\NormalTok{p_hombres <-}\StringTok{ }\NormalTok{n_hombres}\OperatorTok{/}\NormalTok{n_registros}
\KeywordTok{print}\NormalTok{(p_hombres)}
\end{Highlighting}
\end{Shaded}

\begin{verbatim}
## [1] 0.5
\end{verbatim}

\begin{Shaded}
\begin{Highlighting}[]
\CommentTok{# Probabilidad de mujeres}
\NormalTok{p_mujeres =}\StringTok{ }\NormalTok{n_mujeres}\OperatorTok{/}\NormalTok{n_registros}
\KeywordTok{print}\NormalTok{(p_mujeres)}
\end{Highlighting}
\end{Shaded}

\begin{verbatim}
## [1] 0.5
\end{verbatim}

\begin{Shaded}
\begin{Highlighting}[]
\CommentTok{# Media de cada variable por sexo}
\NormalTok{data_mean <-}\StringTok{ }\NormalTok{data }\OperatorTok\StringTok{ }
\StringTok{  }\KeywordTok{group_by}\NormalTok{(sexo) }\OperatorTok\StringTok{ }
\StringTok{  }\KeywordTok{summarise}\NormalTok{(}
    \DataTypeTok{peso =} \KeywordTok{mean}\NormalTok{(peso),}
    \DataTypeTok{altura =} \KeywordTok{mean}\NormalTok{(altura),}
    \DataTypeTok{tamanio_pie =} \KeywordTok{mean}\NormalTok{(tamanio_pie)}
\NormalTok{    )}
\KeywordTok{print}\NormalTok{(data_mean)}
\end{Highlighting}
\end{Shaded}

\begin{verbatim}
## # A tibble: 2 x 4
##   sexo    peso altura tamanio_pie
##   <fct>  <dbl>  <dbl>       <dbl>
## 1 hombre  176.   5.86       11.2 
## 2 mujer   132.   5.42        7.50
\end{verbatim}

\begin{Shaded}
\begin{Highlighting}[]
\CommentTok{# Varianza de cada variable por sexo}
\NormalTok{data_var <-}\StringTok{ }\NormalTok{data }\OperatorTok\StringTok{ }
\StringTok{  }\KeywordTok{group_by}\NormalTok{(sexo) }\OperatorTok\StringTok{ }
\StringTok{  }\KeywordTok{summarise}\NormalTok{(}
    \DataTypeTok{peso =} \KeywordTok{var}\NormalTok{(peso),}
    \DataTypeTok{altura =} \KeywordTok{var}\NormalTok{(altura),}
    \DataTypeTok{tamanio_pie =} \KeywordTok{var}\NormalTok{(tamanio_pie)}
\NormalTok{    )}
\KeywordTok{print}\NormalTok{(data_var)}
\end{Highlighting}
\end{Shaded}

\begin{verbatim}
## # A tibble: 2 x 4
##   sexo    peso altura tamanio_pie
##   <fct>  <dbl>  <dbl>       <dbl>
## 1 hombre  123. 0.0350       0.917
## 2 mujer   558. 0.0972       1.67
\end{verbatim}

\subsubsection{Calculamos la Probabilidad y Varianza de
hombre}\label{calculamos-la-probabilidad-y-varianza-de-hombre}

\begin{Shaded}
\begin{Highlighting}[]
\CommentTok{# Media para hombre}
\NormalTok{mean_hombres <-}\StringTok{ }\NormalTok{data_mean }\OperatorTok\StringTok{ }
\StringTok{  }\KeywordTok{filter}\NormalTok{(sexo}\OperatorTok{==}\StringTok{'hombre'}\NormalTok{) }
\KeywordTok{print}\NormalTok{(mean_hombres)}
\end{Highlighting}
\end{Shaded}

\begin{verbatim}
## # A tibble: 1 x 4
##   sexo    peso altura tamanio_pie
##   <fct>  <dbl>  <dbl>       <dbl>
## 1 hombre  176.   5.86        11.2
\end{verbatim}

\begin{Shaded}
\begin{Highlighting}[]
\CommentTok{#Varianza para hombre}
\NormalTok{var_hombres <-}\StringTok{ }\NormalTok{data_var }\OperatorTok\StringTok{ }
\StringTok{  }\KeywordTok{filter}\NormalTok{(sexo}\OperatorTok{==}\StringTok{'hombre'}\NormalTok{) }
\KeywordTok{print}\NormalTok{(var_hombres)}
\end{Highlighting}
\end{Shaded}

\begin{verbatim}
## # A tibble: 1 x 4
##   sexo    peso altura tamanio_pie
##   <fct>  <dbl>  <dbl>       <dbl>
## 1 hombre  123. 0.0350       0.917
\end{verbatim}

\subsubsection{Calculamos la Probabilidad y Varianza de
mujer}\label{calculamos-la-probabilidad-y-varianza-de-mujer}

\begin{Shaded}
\begin{Highlighting}[]
\CommentTok{# Media para muejer}
\NormalTok{mean_mujeres <-}\StringTok{ }\NormalTok{data_mean }\OperatorTok\StringTok{ }
\StringTok{  }\KeywordTok{filter}\NormalTok{(sexo}\OperatorTok{==}\StringTok{'mujer'}\NormalTok{) }
\KeywordTok{print}\NormalTok{(mean_hombres)}
\end{Highlighting}
\end{Shaded}

\begin{verbatim}
## # A tibble: 1 x 4
##   sexo    peso altura tamanio_pie
##   <fct>  <dbl>  <dbl>       <dbl>
## 1 hombre  176.   5.86        11.2
\end{verbatim}

\begin{Shaded}
\begin{Highlighting}[]
\CommentTok{#Varianza para mujer}
\NormalTok{var_mujeres <-}\StringTok{ }\NormalTok{data_var }\OperatorTok\StringTok{ }
\StringTok{  }\KeywordTok{filter}\NormalTok{(sexo}\OperatorTok{==}\StringTok{'mujer'}\NormalTok{) }
\KeywordTok{print}\NormalTok{(var_hombres)}
\end{Highlighting}
\end{Shaded}

\begin{verbatim}
## # A tibble: 1 x 4
##   sexo    peso altura tamanio_pie
##   <fct>  <dbl>  <dbl>       <dbl>
## 1 hombre  123. 0.0350       0.917
\end{verbatim}

\subsubsection{Función para calcular la probabildad de X dado
Y}\label{funcion-para-calcular-la-probabildad-de-x-dado-y}

\begin{Shaded}
\begin{Highlighting}[]
\NormalTok{p_x_d_y <-}\StringTok{ }\ControlFlowTok{function}\NormalTok{(x, mean_x, var_y)\{}
  \KeywordTok{return}\NormalTok{ ((}\DecValTok{1}\OperatorTok{/}\NormalTok{(}\KeywordTok{sqrt}\NormalTok{(var_y)}\OperatorTok{*}\StringTok{ }\KeywordTok{sqrt}\NormalTok{(}\DecValTok{2}\OperatorTok{*}\NormalTok{pi)))}\OperatorTok{*}\StringTok{ }\KeywordTok{exp}\NormalTok{(}\OperatorTok{-}\DecValTok{1}\OperatorTok{*}\NormalTok{((x}\OperatorTok{-}\NormalTok{mean_x)}\OperatorTok{^}\DecValTok{2}\OperatorTok{/}\NormalTok{(}\DecValTok{2}\OperatorTok{*}\NormalTok{var_y))))}
\NormalTok{\}}
\end{Highlighting}
\end{Shaded}

\subsubsection{Calculamos la Probabilidad de que sea
Hombre}\label{calculamos-la-probabilidad-de-que-sea-hombre}

\begin{Shaded}
\begin{Highlighting}[]
\NormalTok{condicion_hombre <-}\StringTok{ }\NormalTok{p_hombres }\OperatorTok{*}\StringTok{ }\KeywordTok{p_x_d_y}\NormalTok{(persona}\OperatorTok{$}\NormalTok{altura,mean_hombres}\OperatorTok{$}\NormalTok{altura,var_hombres}\OperatorTok{$}\NormalTok{altura) }\OperatorTok{*}\StringTok{ }\KeywordTok{p_x_d_y}\NormalTok{(persona}\OperatorTok{$}\NormalTok{peso,mean_hombres}\OperatorTok{$}\NormalTok{peso,var_hombres}\OperatorTok{$}\NormalTok{peso) }\OperatorTok{*}\StringTok{ }\KeywordTok{p_x_d_y}\NormalTok{(persona}\OperatorTok{$}\NormalTok{tamanio_pies,mean_hombres}\OperatorTok{$}\NormalTok{tamanio_pie,var_hombres}\OperatorTok{$}\NormalTok{tamanio_pie)}
\KeywordTok{print}\NormalTok{(condicion_hombre)}
\end{Highlighting}
\end{Shaded}

\begin{verbatim}
## [1] 6.197072e-09
\end{verbatim}

\subsubsection{Calculamos la Probabilidad de que sea
Mujer}\label{calculamos-la-probabilidad-de-que-sea-mujer}

\begin{Shaded}
\begin{Highlighting}[]
\NormalTok{condicion_mujer <-}\StringTok{ }\NormalTok{p_mujeres }\OperatorTok{*}\StringTok{ }\KeywordTok{p_x_d_y}\NormalTok{(persona}\OperatorTok{$}\NormalTok{altura,mean_mujeres}\OperatorTok{$}\NormalTok{altura,var_mujeres}\OperatorTok{$}\NormalTok{altura) }\OperatorTok{*}\StringTok{ }\KeywordTok{p_x_d_y}\NormalTok{(persona}\OperatorTok{$}\NormalTok{peso,mean_mujeres}\OperatorTok{$}\NormalTok{peso,var_mujeres}\OperatorTok{$}\NormalTok{peso) }\OperatorTok{*}\StringTok{ }\KeywordTok{p_x_d_y}\NormalTok{(persona}\OperatorTok{$}\NormalTok{tamanio_pies,mean_mujeres}\OperatorTok{$}\NormalTok{tamanio_pie,var_mujeres}\OperatorTok{$}\NormalTok{tamanio_pie)}
\KeywordTok{print}\NormalTok{(condicion_mujer)}
\end{Highlighting}
\end{Shaded}

\begin{verbatim}
## [1] 0.0005377909
\end{verbatim}

\subsubsection{Decidimos cual es la probabilidad
mayor}\label{decidimos-cual-es-la-probabilidad-mayor}

\paragraph{Decimos el resultado}\label{decimos-el-resultado}

\begin{Shaded}
\begin{Highlighting}[]
\ControlFlowTok{if}\NormalTok{(condicion_hombre}\OperatorTok{>}\NormalTok{condicion_mujer)\{}
  \KeywordTok{print}\NormalTok{(}\StringTok{'Hombre'}\NormalTok{)}
\NormalTok{\}}\ControlFlowTok{else}\NormalTok{\{}
  \KeywordTok{print}\NormalTok{(}\StringTok{'Mujer'}\NormalTok{)}
\NormalTok{\}}
\end{Highlighting}
\end{Shaded}

\begin{verbatim}
## [1] "Mujer"
\end{verbatim}


\end{document}
